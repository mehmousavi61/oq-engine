Fault sources in the \gls{acr:oqe} are classified according to the method
adopted to distribute ruptures over the fault surface. Two options are
currently supported:

\begin{itemize}

    \item With the first option, ruptures with a surface lower than the
    whole fault surface are floated so as to cover as much as possible
    homogeneously the fault surface. This model is compatible with all the
    supported magnitude-frequency distributions.

    \item With the second option, ruptures always fill the entire fault
    surface. This model is compatible with magnitude-frequency
    distributions similar to a characteristic model (\`{a} la
    \cite{schwartz1984}).

\end{itemize}

In this subsection we discuss the different fault source types that
support floating ruptures. In the next subsection we will illustrate the fault
typology available to model a characteristic rupturing behaviour.



\subsubsection{Simple faults}
\label{desc_simple_fault}
\index{Source type!fault!simple geometry}
\index{Simple fault|see{Source type}}

Simple Faults are the most common source type used to model shallow faults;
the ``simple'' adjective relates to the geometry description of the source
which is obtained by projecting the fault trace (i.e. a polyline) along a
characteristic dip direction.

The parameters used to create an instance of this source type are described
in the following paragraph.

\paragraph{Source data}

\begin{itemize}

    \item A horizontal \gls{faulttrace} (usually a polyline). It is a list of
    longitude-latitude tuples [degrees].

    \item A \gls{frequencymagnitudedistribution}

    \item A \gls{msr}

    \item A representative value of the dip angle (specified following
    the Aki-Richards convention; see \citet{aki2002}) [degrees]

    \item Rake angle (specified following the Aki-Richards convention;
    see \citet{aki2002}) [degrees]

    \item Upper and lower depth values limiting the seismogenic interval [km]

\end{itemize}

For near-fault probabilistic seismic hazard analysis, two additional
parameters are needed for characterising seismic sources:

\begin{itemize}

    \item A hypocentre list. It is a list of the possible hypocentral
    positions, and the corresponding weights, e.g., alongStrike="0.25"
    downDip="0.25" weight="0.25". Each hypocentral position is defined in
    relative terms using as a reference the upper left corner of the rupture
    and by specifying the fraction of rupture length and rupture width.

    \item A slip list. It is a list of the possible rupture slip directions
    [degrees], and their corresponding weights. The angle describing each slip
    direction is measured counterclockwise using the fault strike direction as
    reference.

\end{itemize}

In near-fault PSHA calculations, the hypocentre list and the slip list are
mandatory. The weights in each list must always sum to one. The available GMPE
which currently supports the near-fault directivity PSHA calculation in OQ-
engine is the ChiouYoungs2014NearFaultEffect GMPE developed  by
\citet{chiou2014update} (associated with an \texttt{Active Shallow Crust}
tectonic region type).

We provide two examples of simple fault source files. The first is an
excerpt of an xml file used to describe the properties of a simple fault
source and the second example shows the excerpt of an xml file used to
describe the properties of a simple fault source that can be used to perform a PSHA calculation taking into account directivity effects.

\begin{listing}[htbp]
  \inputminted[firstline=1,firstnumber=1,fontsize=\footnotesize,frame=single,linenos,bgcolor=lightgray]{xml}{oqum/hazard/verbatim/input_simple_fault.xml}
  \caption{Example simple fault}
  \label{lst:example_simple_fault}
\end{listing}
%\input{oqum/hazard/verbatim/input_simple_fault}
%\label{example_incremental_mfd}

%Below is an excerpt of a simple fault source xml file for near-fault directivity PSHA calculations:

%\begin{listing}[htbp]
\inputminted[firstline=1,firstnumber=1,fontsize=\footnotesize,frame=single,linenos,bgcolor=lightgray]{xml}{oqum/hazard/verbatim/input_simple_fault_directivity.xml}
\captionof{listing}{Example simple fault with added information to model directivity\label{lst:example_simple_fault_directivity}}

%\end{listing}

%\input{oqum/hazard/verbatim/input_simple_fault_directivity}

%As with the previous examples, the red text highlights the parameters used to specify the source geometry, the parameters in green describe the rupture mechanism, the text in blue describes the magnitude-frequency distribution and the gray text describes the rupture properties.



\subsubsection{Complex faults}
\label{desc_complex_fault}
\index{Source type!fault!complex geometry}
\index{Complex fault|see{Source type}}

A complex fault differs from simple fault just by the way the geometry of the
fault surface is defined and the fault surface is later created. The input
parameters used to describe complex faults are, for the most part, the same
used to describe the simple fault typology.

In the case of complex faults, the dip angle is not requested while the fault
trace is substituted by two fault edges limiting the top and bottom of the fault
surface. Additional curves lying over the fault surface can be specified to
complement and refine the description of the fault surface geometry.
Unlike the simple fault these edges are not required to be horizontal
and may vary in elevation, i.e. the upper edge may represent the intersection 
between the exposed fault trace and the topographic surface, where positive values
indicate below sea level, and negative values indicate above sea level.

Usually, we use complex faults to model intraplate megathrust faults such as
the big subduction structures active in the Pacific (Sumatra, South America,
Japan) but this source typology can be used also to create - for example -
listric fault sources with a realistic geometry.

\inputminted[firstline=1,firstnumber=1,fontsize=\footnotesize,frame=single,linenos,bgcolor=lightgray]{xml}{oqum/hazard/verbatim/input_complex_fault.xml}
\captionof{listing}{Example complex fault \label{lst:example_complex_fault}}

%\input{oqum/hazard/verbatim/input_complex_fault}

As with the previous examples, the red text highlights the parameters used to
specify the source geometry, the parameters in green describe the rupture
mechanism, the text in blue describes the magnitude-frequency distribution and
the gray text describes the rupture properties.
